\documentclass{article}
%\documentclass[UTF8]{ctexart}
\usepackage{ctex}

% if you need to pass options to natbib, use, e.g.:
%     \PassOptionsToPackage{numbers, compress}{natbib}
% before loading neurips_2024


% ready for submission
\usepackage[preprint]{neurips_2024}


% to compile a preprint version, e.g., for submission to arXiv, add add the
% [preprint] option:
%     \usepackage[preprint]{neurips_2024}


% to compile a camera-ready version, add the [final] option, e.g.:
%     \usepackage[final]{neurips_2024}


% to avoid loading the natbib package, add option nonatbib:
%    \usepackage[nonatbib]{neurips_2024}


\usepackage[utf8]{inputenc} % allow utf-8 input
\usepackage[T1]{fontenc}    % use 8-bit T1 fonts
\usepackage{hyperref}       % hyperlinks
\usepackage{url}            % simple URL typesetting
\usepackage{booktabs}       % professional-quality tables
\usepackage{amsfonts}       % blackboard math symbols
\usepackage{nicefrac}       % compact symbols for 1/2, etc.
\usepackage{microtype}      % microtypography
\usepackage{xcolor}         % colors


\title{文本表征学习HW2}


% The \author macro works with any number of authors. There are two commands
% used to separate the names and addresses of multiple authors: \And and \AND.
%
% Using \And between authors leaves it to LaTeX to determine where to break the
% lines. Using \AND forces a line break at that point. So, if LaTeX puts 3 of 4
% authors names on the first line, and the last on the second line, try using
% \AND instead of \And before the third author name.


\author{
  罗浩铭 \\
  PB21030838\\
  中国科学技术大学\ 安徽合肥 230026 \\
  \texttt{mzfslhm@mail.ustc.edu.cn} \\
  % examples of more authors
  % \And
  % Coauthor \\
  % Affiliation \\
  % Address \\
  % \texttt{email} \\
  % \AND
  % Coauthor \\
  % Affiliation \\
  % Address \\
  % \texttt{email} \\
  % \And
  % Coauthor \\
  % Affiliation \\
  % Address \\
  % \texttt{email} \\
  % \And
  % Coauthor \\
  % Affiliation \\
  % Address \\
  % \texttt{email} \\
}


\begin{document}


\maketitle


\begin{abstract}
  The abstract paragraph should be indented \nicefrac{1}{2}~inch (3~picas) on
  both the left- and right-hand margins. Use 10~point type, with a vertical
  spacing (leading) of 11~points.  The word \textbf{Abstract} must be centered,
  bold, and in point size 12. Two line spaces precede the abstract. The abstract
  must be limited to one paragraph.
\end{abstract}

\section{项目实现}

% 简要描述处理数据和实现模型的过程,其中使用了哪些数据结构,遇到了哪些问题,是如何解决的
本项目主要基于\href{https://code.google.com/archive/p/word2vec/source/default/source}{谷歌的word2vec实现}来完成。

\subsection{数据处理}

我们使用enwiki8数据集作为训练数据。由于实验所用的代码仅靠空白字符(空格、换行等)来完成分词,
而enwiki8为XML格式套HTML格式,其中的内容混杂了大量与单词语义无关的格式符,因此我们需要对数据进行预处理。

我们去除了XML标签、HTML标签、HTML格式符、表格、图片等内容,只保留了文本内容。我们还在所有标点符号左右都添加了空格,并去除了其余所有特殊符号。
我们又将所有字母转为小写(但未做进一步的词干提取)。最终,我们得到了纯净的文本内容,用于后续实验。
(该处理主要借用了\texttt{demo-train-big-model-v1.sh}中的部分脚本代码来完成)


\subsection{模型实现}


\section{训练参数设置}


\section{实验结果与分析}

\subsection{实验结果}
% 四种方法训练得到的词向量的性能结果

\subsection{结果分析}
% 结合你对四种方法结果的观察,比较分析HS与NS,以及CBOW和SG

\end{document}