\documentclass{article}
%\documentclass[UTF8]{ctexart}
\usepackage{ctex}

% if you need to pass options to natbib, use, e.g.:
%     \PassOptionsToPackage{numbers, compress}{natbib}
% before loading neurips_2024


% ready for submission
\usepackage[preprint]{neurips_2024}


% to compile a preprint version, e.g., for submission to arXiv, add add the
% [preprint] option:
%     \usepackage[preprint]{neurips_2024}


% to compile a camera-ready version, add the [final] option, e.g.:
%     \usepackage[final]{neurips_2024}


% to avoid loading the natbib package, add option nonatbib:
%    \usepackage[nonatbib]{neurips_2024}


\usepackage[utf8]{inputenc} % allow utf-8 input
\usepackage[T1]{fontenc}    % use 8-bit T1 fonts
\usepackage{hyperref}       % hyperlinks
\usepackage{url}            % simple URL typesetting
\usepackage{booktabs}       % professional-quality tables
\usepackage{amsfonts}       % blackboard math symbols
\usepackage{nicefrac}       % compact symbols for 1/2, etc.
\usepackage{microtype}      % microtypography
\usepackage{xcolor}         % colors

\usepackage{listings}
\usepackage{color}

\definecolor{dkgreen}{rgb}{0,0.6,0}
\definecolor{gray}{rgb}{0.5,0.5,0.5}
\definecolor{mauve}{rgb}{0.58,0,0.82}

\lstset{frame=tb,
  language=Python,
  aboveskip=3mm,
  belowskip=3mm,
  showstringspaces=false,
  columns=flexible,
  basicstyle={\small\ttfamily},
  numbers=none,
  numberstyle=\tiny\color{gray},
  keywordstyle=\color{blue},
  commentstyle=\color{dkgreen},
  stringstyle=\color{mauve},
  breaklines=true,
  breakatwhitespace=true,
  tabsize=3
}


\title{文本表征学习HW4: Transformer}


% The \author macro works with any number of authors. There are two commands
% used to separate the names and addresses of multiple authors: \And and \AND.
%
% Using \And between authors leaves it to LaTeX to determine where to break the
% lines. Using \AND forces a line break at that point. So, if LaTeX puts 3 of 4
% authors names on the first line, and the last on the second line, try using
% \AND instead of \And before the third author name.


\author{
  罗浩铭 \\
  PB21030838\\
  中国科学技术大学\ 安徽合肥 230026 \\
  \texttt{mzfslhm@mail.ustc.edu.cn} \\
  % examples of more authors
  % \And
  % Coauthor \\
  % Affiliation \\
  % Address \\
  % \texttt{email} \\
  % \AND
  % Coauthor \\
  % Affiliation \\
  % Address \\
  % \texttt{email} \\
  % \And
  % Coauthor \\
  % Affiliation \\
  % Address \\
  % \texttt{email} \\
  % \And
  % Coauthor \\
  % Affiliation \\
  % Address \\
  % \texttt{email} \\
}


\begin{document}


\maketitle


\begin{abstract}
  本次实验完成了基于Transformer(BERT)模型的情感分类任务训练的全过程,包括数据处理、模型训练、性能评估,并对比分析了不同训练设定的性能。
  实验结果显示,
  Transformer架构是当今所有性能惊艳的大模型的基石。本次实验中对Transformer架构的探索,将为我们深入学习深度学习在NLP领域的应用提供更好的基础。
\end{abstract}

% 概述:使用训练数据以及Transformer模型训练句子向量并对其性能进行评价,对比分析不同设定的性能,完成并提交1-2页的实验报告

% 使用IMDB情感分析数据集,使用以下两种配置,训练Transformer模型:
% - 向量维度设置为100,其他参数自定
% - 向量维度设置为200,保持与上个模型的参数一致
% - 针对情感分析任务,比较分析测试集上性能差异
% - 情感分析任务方法:自定义句向量输出方式,使用逻辑回归分类


% 实验报告应包含:
% - 两种设定得到的Transformer模型在IMDB测试集上性能(准确率)结果
% - 结合你对两种模型结果的观察,比较分析向量维度与其他因素(如参数设定等)对结果的影响
% - 简要描述处理数据和实现模型的过程,其中使用了哪些数据结构,遇到了哪些问题,是如何解决的

\section{项目实现}
我们使用了Huggingface Datasets库及Huggingface Transformers库来实现我们的项目。
\subsection{数据处理}
对于MLM预训练数据集,我们将原数据集中的文本塞入\verb|LineByLineTextDataset|类中来构建自回归文本数据集(块大小最大为128),
然后使用\verb|DataCollatorForLanguageModeling|将其中15\%的token随机mask掉,以此来构建MLM预训练数据集。

对于分类数据集,我们使用了Huggingface Datasets库来加载IMDB数据集,然后使用中的Tokenizer类来对文本进行分词。
分词过程中,我们限定每个文本的分词长度为512,对token长度超过512的文本进行,并对长度不足的文本填充\verb|[PAD]| token并加上相应的attention mask。


\subsection{模型实现}
由于BERT模型相当于是原生的Transformer模型的Encoder部分加上特殊的训练方式,因此我们的实验中主要使用了BERT式的模型来完成MLM预训练任务。

我们的模型以Huggingface中的\verb|prajjwal1/bert-tiny|模型为基础,\textbf{从头}对tiny-BERT模型进行训练。
对于模型架构,我们仅仅修改了hidden_size为实验要求的100和200,并保持intermediate_size是其四倍,其他参数不变。
我们的分词器直接借用了\verb|prajjwal1/bert-tiny|模型的分词器。

我们使用了Transformers库中的\verb|BertForMaskedLM|类来实现MLM预训练任务,使用上述自回归文本数据集来训练模型。其训练参数如下:
\begin{itemize}
  \item epoch数为10
  \item batch size为64
  \item 学习率为3e-4
  \item 使用AdamW优化器
  \item 使用cosine学习率调度器
  \item 使用warmup比例为0.1
  \item 使用weight decay为0.01
\end{itemize}
其余参数为Transformers库中\verb|TrainingArguments|类的默认参数。

最后,我们使用\verb|statsmodels|库中的逻辑回归类\verb|Logit|来完成文本情感分类任务,
使用训练好的tiny-BERT模型的\verb|[CLS]|位置对应输出来作为文本的向量表示。



\section{实验结果与分析}

我们使用的向量维度训得tiny-BERT模型,此后基于训得的tiny-BERT模型来训练文本情感分类的逻辑回归模型,将逻辑回归模型在测试集上的效果作为实验结果。
此外,我们还比较了预训练的\verb|prajjwal1/bert-tiny|模型以及当前先进的Embedding模型\verb|nomic-ai/nomic-embed-text-v1.5|模型(2024年2月份提出的近SOTA模型,基于BERT架构,参数量137M)在IMDB测试集上的性能。

得到的实验结果如表格\ref{tab:results}所示:
\begin{table}[htbp]
  \caption{各种模型及参数设置下得到的Doc向量性能}
  \label{tab:results}
  \vspace{5pt}
  \centering
  \begin{tabular}{cccc}
    \toprule
    逻辑回归准确率          & Train   & Test    \\
    \midrule
    BERT (hidden_size=100)  & $0.710$ & $0.697$ \\
    \midrule
    BERT (hidden_size=200)  & $0.718$ & $0.709$ \\
    \midrule
    \verb|bert-tiny| & $0.766$ & $0.759$ \\
    \midrule
    \verb|nomic-embed-text-v1.5| & $0.945$ & $0.933$ \\
    \bottomrule
  \end{tabular}
\end{table}

从表格\ref{tab:results}中我们可以看出:
\begin{itemize}
  \item tiny-BERT模型在IMDB测试集上的性能随着hidden size的增大而有所提升,但提升并不明显。
  \item tiny-BERT模型的预训练效果不如Doc2Vec模型,这可能是因为tiny-BERT模型的归纳偏置较少,且训练数据量不足导致的。
  \item 在进行较为充分的预训练后,tiny-BERT模型的效果理论上可以提升约5\%,还是不如Doc2Vec模型,这说明Doc2Vec中对Paragraph vector进行迭代优化的过程可能是有很大用处的。
  \item 我们测试了先进的\verb|nomic-embed-text-v1.5|模型,其性能远远超过了tiny-BERT模型以及Doc2Vec模型。
\end{itemize}

% 3 epochs: Train accuracy:  0.64, Test accuracy:  0.62
% 10 epochs: Train accuracy:  0.67796, Test accuracy:  0.64824

% From now on, batch_size=256
% 5 epochs: Train accuracy:  0.698, Test accuracy:  0.65648

% pretrained mlm (hidden_size=128, batch_size=64, lr=3e-4): Train accuracy:  0.8094, Test accuracy:  0.78704

% MLM:
% 5 epochs: Train accuracy:  0.8094, Test accuracy:  0.78704
% 10 epochs, final params: Train accuracy:  0.7098, Test accuracy:  0.6968
% 10 epochs, final params, dim=200: Train accuracy:  0.718, Test accuracy:  0.70884

% pretrain: Train accuracy:  0.76628, Test accuracy:  0.759
% pretrain, average token: Train accuracy:  0.76948, Test accuracy:  0.75916

% nomic: Train accuracy:  0.945, Test accuracy:  0.93304

\end{document}

