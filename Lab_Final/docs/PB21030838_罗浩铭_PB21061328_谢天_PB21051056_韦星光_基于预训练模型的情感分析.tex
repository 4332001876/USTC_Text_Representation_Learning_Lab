\documentclass{article}
%\documentclass[UTF8]{ctexart}
\usepackage{ctex}
% \usepackage{verbatim}  % comment env

% if you need to pass options to natbib, use, e.g.:
%     \PassOptionsToPackage{numbers, compress}{natbib}
% before loading neurips_2024


% ready for submission
\usepackage[preprint]{neurips_2024}


% to compile a preprint version, e.g., for submission to arXiv, add add the
% [preprint] option:
%     \usepackage[preprint]{neurips_2024}


% to compile a camera-ready version, add the [final] option, e.g.:
%     \usepackage[final]{neurips_2024}


% to avoid loading the natbib package, add option nonatbib:
%    \usepackage[nonatbib]{neurips_2024}


\usepackage[utf8]{inputenc} % allow utf-8 input
\usepackage[T1]{fontenc}    % use 8-bit T1 fonts
\usepackage{hyperref}       % hyperlinks
\usepackage{url}            % simple URL typesetting
\usepackage{booktabs}       % professional-quality tables
\usepackage{amsfonts}       % blackboard math symbols
\usepackage{nicefrac}       % compact symbols for 1/2, etc.
\usepackage{microtype}      % microtypography
\usepackage{xcolor}         % colors

\usepackage{listings}
\usepackage{color}

\definecolor{dkgreen}{rgb}{0,0.6,0}
\definecolor{gray}{rgb}{0.5,0.5,0.5}
\definecolor{mauve}{rgb}{0.58,0,0.82}

\lstset{frame=tb,
  language=Python,
  aboveskip=3mm,
  belowskip=3mm,
  showstringspaces=false,
  columns=flexible,
  basicstyle={\small\ttfamily},
  numbers=none,
  numberstyle=\tiny\color{gray},
  keywordstyle=\color{blue},
  commentstyle=\color{dkgreen},
  stringstyle=\color{mauve},
  breaklines=true,
  breakatwhitespace=true,
  tabsize=3
}


\title{文本表征学习Final: 基于预训练大模型的情感分析}


% The \author macro works with any number of authors. There are two commands
% used to separate the names and addresses of multiple authors: \And and \AND.
%
% Using \And between authors leaves it to LaTeX to determine where to break the
% lines. Using \AND forces a line break at that point. So, if LaTeX puts 3 of 4
% authors names on the first line, and the last on the second line, try using
% \AND instead of \And before the third author name.
\author{
~~{\large 罗浩铭\quad}
~~{\large 谢天\quad}
~~{\large 韦星光\quad}\\
~~{PB21030838}
~~{PB21061328}
~~{PB21051056}\\
中国科学技术大学\ \ \ 安徽合肥\\
{\tt \{mzfslhm, xqs2002, wesustc\}@ustc.mail.edu.cn} \\
}

% \author{
%   罗浩铭 \\
%   PB21030838\\
%   中国科学技术大学\ 安徽合肥 230026 \\
%   \texttt{mzfslhm@mail.ustc.edu.cn} \\
%   \And
%   谢天 \\
%   PB21061328\\
%   中国科学技术大学\ 安徽合肥 230026 \\
%   \texttt{xqs2002@mail.ustc.edu.cn} \\
%   \And
%   韦星光 \\
%   PB21051056\\
%   中国科学技术大学\ 安徽合肥 230026 \\
%   \texttt{wesustc@mail.ustc.edu.cn} \\
% }


\begin{document}


\maketitle


\begin{abstract}
  abstract
\end{abstract}


% 期末作业:基于预训练/大模型的情感分析
% 任务:使用自选模型实现情感分析,完成实验报告
% 数据集:情感分析数据集IMDB

% 需要的内容
% • 动机(包括方法选择、实验设定的原因等)
% • 实验细节(数据统计信息、模型参数等)
% • 实验结果分析(需要附主要实验结果的图表)
% • 讨论(实验中遇到的困难和解决方法等)
% • 组员分工及贡献情况说明
% • 参考文献列表


% 额外说明
% • 可根据实际算力情况仅使用其中部分训练数据(需在报告中写明)
% • 可以使用不同种类预训练模型,包括ChatGPT等大模型
% 加分项
% • 使用方法具有一定的创新性,具备进一步拓展为学术论文的潜力
% • 系统性比较多个预训练模型的效果
% • 使用额外的情感分析数据集比较分析结果
% • 系统全面地分析模型的效果,在主实验讨论分析之外,加入消融实验、错误分析、案例分析等

% • 提交:不超过4页(不含组员分工和参考文献)论文形式的报告
% • 包括:前言、方法(主要工作)、实验设定和结果、分析讨论、结论、组员分工、参考文献列表
% • 作业提交时间:6月23日00:00之前


\section{Introduction}
% • 动机(包括方法选择、实验设定的原因等)
我们对比了以下技术方案的效果:

nomic

预训练模型

CARP prompt

微调

\section{实验设定和结果}
% • 实验细节(数据统计信息、模型参数等)

得到的实验结果如表格\ref{tab:pretrained_results}所示:
\begin{table}[htbp]
  \caption{\small{各种模型及参数设置下得到的Doc向量性能}}
  \label{tab:pretrained_results}
  \centering
  \begin{tabular}{cccc}
    \toprule
    模型                   & Accuracy         \\
    \midrule
    \verb|nomic-embed-text-v1.5| & $0.945$          \\
    \midrule
    \verb|Qwen2-0.5B-Instruct| & $0.745$          \\
    \midrule
    \verb|Qwen2-1.5B-Instruct| & $0.898$          \\
    \midrule
    \verb|Qwen2-7B-Instruct| & $\textbf{0.946}$ \\
    \bottomrule
  \end{tabular}
\end{table}


\section{微调实验}
\subsection{指令微调数据集构造}
\subsection{LoRA微调}
根据论文X给出的实践经验指导,我们合理选择了LoRA的超参数与部分训练参数:
...
\subsection{全量微调}
\section{分析讨论}
% • 实验结果分析(需要附主要实验结果的图表)
% • 讨论(实验中遇到的困难和解决方法等)

\section{结论}


\section{组员分工}
% • 组员分工及贡献情况说明

\section{参考文献}
% • 组员分工及贡献情况说明





\end{document}

